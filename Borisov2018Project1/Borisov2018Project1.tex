\documentclass[12pt, twoside]{article}
\bibliographystyle{plain}
\bibliography{Project1}
\usepackage{jmlda}
\usepackage{hyperref}
%\NOREVIEWERNOTES
\title
    [Прогнозирование направления движения цены биржевых инструментов по новостному потоку ] % Краткое название; не нужно, если полное название влезает в~колонтитул
    {Прогнозирование направления движения цены биржевых инструментов по новостному потоку.}
\author
    {Ахияров~В.,\, Борисов~А.,\, Говоров~И.,\, Дробин~М.,\, Мухитдинова~С.,\, Родионов~В. } % основной список авторов, выводимый в оглавление

\email
    {akhiarov.va@phystech.edu,\, borisov.as@phystech.edu,\, govorov.is@phystech.edu,\, drobin.me@phystech.edu,\, muhitdinova.sm@phystech.edu,\, rodionov.vo@phystech.edu}

\organization
    {МФТИ (ГУ)}
\abstract
    { \textbf{Аннотация}: В работе рассматривается задача классификации направления движения временных рядов. 
Классификация производится с помощью анализа признаков из отчётов 8-K, которые компании обязаны заполнять 
при значительных событиях, таких как банкротство, выбор совета директоров и пр. 
Рассматривается несколько моделей классификации. 
В одних используются только признаки из отчётов, 1-граммы которых встречающиеся более 10 раз. 
В других к предыдущему этапу применяется неотрицательная матричная факторизация (NMF). 
И в последней, ансамбле, объединяются предыдущие подходы путём голосования большинства. 
В качестве прикладной задачи рассматривается задача распознавания направления движения акций по новостям, выраженных 
отчётами 8-K. 
Модели классификации, исследованные в этой 
работе, сравнивается в точности и статистической значимости с простыми моделями, использующими только прогнозируемый показатель доход на акцию 
или использующую только финансовые показатели. 

\bigskip

\textbf{Ключевые слова}:  \emph{метрическая классификация, анализ текстов, классификация 
временных рядов, новостной поток}
}

\begin{document}
\maketitle
%\linenumbers

\section{Введение}

\begin{enumerate}
\item \textbf{Прогнозирование направления движения цены биржевых инструментов по новостному потоку.}
Мотивируемое тем, что флуктуации цен на бирже, сильно зависящие от политической, географической и т.д. обстановок,  интересные не только при скальпинге. Для среднесрочных торгов и инвестиций такие данные так же имеют большую роль, позволяя корректировать вложения. Как правило, крупные изменения в политике, природные катаклизмы и все события которые именяют распределение цен котировок, освещаются в прессе. 

\item Исследование строится вокруг постоянных изменений цен биржевых котировок, новостей, и алгоритма NMF вектора.

\item Требуется на основе большого количество новой информации (предоставляемой в разрозненном текстовом виде)  касающейся компаний, перечисленных на фондовом рынке, предсказать повышение, понижение либо стабилизацию цен на акции, ценные бумаги и т.д. Необходимо разработать модель, которая также учитывает недавнее движение акций, и так называемую “неожиданную прибыль”(отчет о прибылях и убытках компании, значительно отличающийся (в положительном или отрицательном направлении) от ожиданий аналитиков (согласованного прогноза)

\item \textbf{Методы исследования.}
В работе приведены другие, которые как улучшают уже существующие, так и вводят новые методы обработки естественного языка.
Так в Xie et al. (2013) вводится дерево представлений об информации в новостях, в Bollen et al. (2010) использованы данные из Twitter'a.
Bar-Haim et al. (2011) распознают лучших экспертов-инвесторов, а Leinweber and Sisk (2011) исследуют влияние новостей и времени усвоения новостей в событийной торговле.
В Kogan et al. (2009) приводится предсказание риска по финансовым отчётам и в Engelberg (2008) - закономерность о том, что лингвистическая информация (возможно из-за когнитивной нагрузки при обработке) имеет более долгосрочную предсказуемость цен, нежели количественная информация.

\item \textbf{Решаемая в данной работе задача.}
Построить и исследовать модель прогнозирования направления движения цены. Задано множество новостей S и множество временных меток T, соответствующих времени публикации новостей из S. 2. Временной ряд P, соответствующий значению цены биржевого инструмента, и временной ряд V, соответствующий объему продаж по данному инструменту, за период времени T'. 3. Множество T является подмножеством периода времени T'. 4. Временные отрезки w=[w0, w1], l=[l0, l1], d=[d0, d1], где w0 < w1=l0 < l1=d0 < d1. Требуется спрогнозировать направление движения цены биржевого инструмента в момент времени t=d0 по новостям, вышедшим в период w.

\item \textbf{Предлагаемое решение.}
8К - отчеты компаний об их внутренних событиях. Данная отчетность выходит строго в период между закрытием торгов в один день и их открытием на следующий день.
Из отчета 8К убираются все HTML-теги, таблицы и прочее.
Используется метод NMF вектора.
Вычитается из цен сегодняшнего открытия торгов вчерашние цены закрытия торгов с поправкой на индекс.
Берется текст отчета 8К и на выходе нейронной сети функция, принимающая три значения :
*UP-цена открытия следующего дня больше на 1+% от предыдущего дня - “изменение индекса”
*DOWN- цена открытия следующего дня меньше на 1+% от предыдущего дня - “изменение индекса” 
*STAY - цена открытия следующего дня в пределах +/-1%  от предыдущего дня - “изменение индекса”

\item \href{https://drive.google.com/file/d/12KsFJNEADfXYLlV0Ler19A1E9N1pGa3-/view?usp=sharing}{\color{blue}{Работа, описывающая наиболее близкое решение}} 

\item \textbf{Плюсы метода:}
Большой объем данных
Он более доступен небольшим инвесторам, чем real-time trading tools, которыми пользуются большие трейдинговые компании
Он показывает accuracy на 10% больше, baseline, который использует только финансовые фичи(см. Статью в пункте 7) 
смотрят “изменение цены”-”изменение индекса” => чистое влияние
все дивидендные гэпы убирали 

\textbf{Минусы:}
Исследование проведено на рынке США, где отчеты выходят не в торговое время => вся информация отражается мгновенно в цене акции от открытии
результаты не имеют значения на практике => невозможно извлечь финансовую прибыль
Метод не улавливает такие эффекты, как: slippage, transaction costs, borrowing costs

\item -
\item Эксперимент будет проведен на финансовых данных: данные о котировках (с интервалом в один тик) нескольких финансовых инструментов (GAZP, SBER, VTBR, LKOH) за 2 квартал 2017 года с сайта Finam.ru; для каждой точки ряда известны дата, время, цена и объем. И на текстовых данных: экономические новости за 2 квартал 2017 года от компании Форексис; каждая новость является отдельным html файлом.
\end{enumerate}

\begin{thebibliography}{9}
\bibitem{journals/ijon/HuTZW18}
Hongping Hu, Li Tang, Shuhua Zhang, Haiyan Wang (2018) \emph{Predicting the direction of stock markets using optimized neural networks with Google Trends}, Neurocomputing.

\bibitem{conf/clef/KuznetsovMKS16}
Mikhail Kuznetsov, Anastasia Motrenko, Rita Kuznetsova, Vadim Strijov (2016) \emph{Methods for Intrinsic Plagiarism Detection and Author Diarization}, CLEF (Working Notes).

\bibitem{conf/lrec/LeeSMJ14}
Heeyoung Lee, Mihai Surdeanu, Bill MacCartney, Dan Jurafsky (2014) \emph{On the Importance of Text Analysis for Stock Price Prediction}, Proceedings of the Ninth International Conference on Language Resources and Evaluation.

\bibitem{journals/corr/abs-1711-04154}
Anna Potapenko, Artem Popov, Konstantin Vorontsov (2017) \emph{Interpretable probabilistic embeddings: bridging the gap between topic models and neural networks}, CoRR.

\bibitem{Sun2016}
Andrew Sun, Michael Lachanski, Frank J. Fabozzi (2016) \emph{Trade the tweet: Social media text mining and sparse matrix factorization for stock market prediction}, International Review of Financial Analysis.

\bibitem{Usmanova2018TimeSeriesCorrelation}
Усманова К. Р., Кудияров С. П., Мартышкин Р. В., Замковои А. А., Стрижов В. В. (2018) \emph{Анализ зависимостей между показателями при прогнозировании объема грузоперевозок}, Системы и средства информатики.
\end{thebibliography}




\end{document}